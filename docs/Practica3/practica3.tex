\documentclass[11pt,a4paper]{article}
\usepackage[spanish,es-nodecimaldot]{babel}	% Utilizar español
\usepackage[utf8]{inputenc}					% Caracteres UTF-8
\usepackage{graphicx}						% Imagenes
\usepackage[hidelinks]{hyperref}			% Poner enlaces sin marcarlos en rojo
\usepackage{fancyhdr}						% Modificar encabezados y pies de pagina
\usepackage{float}							% Insertar figuras
\usepackage[textwidth=390pt]{geometry}		% Anchura de la pagina
\usepackage[nottoc]{tocbibind}				% Referencias (no incluir num pagina indice en Indice)
\usepackage{enumitem}						% Permitir enumerate con distintos simbolos
\usepackage[T1]{fontenc}					% Usar textsc en sections
\usepackage{amsmath}						% Símbolos matemáticos
\usepackage{listings}
\usepackage{color}

 
\definecolor{codegreen}{rgb}{0,0.6,0}
\definecolor{codegray}{rgb}{0.5,0.5,0.5}
\definecolor{codepurple}{rgb}{0.58,0,0.82}
\definecolor{backcolour}{rgb}{0.99,0.99,0.99}
 
\lstdefinestyle{mystyle}{
    backgroundcolor=\color{backcolour},   
    commentstyle=\color{codegreen},
    keywordstyle=\color{magenta},
    numberstyle=\tiny\color{codegray},
    stringstyle=\color{codepurple},
    basicstyle=\footnotesize,
    breakatwhitespace=false,         
    breaklines=true,                 
    captionpos=b,                    
    keepspaces=true,                 
    numbers=left,                    
    numbersep=5pt,                  
    showspaces=false,                
    showstringspaces=false,
    showtabs=false,                  
    tabsize=2
}
 
\lstset{style=mystyle, language=Java}

% Comando para poner el nombre de la asignatura
\newcommand{\asignatura}{Técnicas de los Sistemas Inteligentes}
\newcommand{\autor}{José María Sánchez Guerrero}
\newcommand{\titulo}{Práctica 3}
\newcommand{\subtitulo}{Planificación Clásica (PDDL)}

% Configuracion de encabezados y pies de pagina
\pagestyle{fancy}
\lhead{\autor{}}
\rhead{\asignatura{}}
\lfoot{Grado en Ingeniería Informática}
\cfoot{}
\rfoot{\thepage}
\renewcommand{\headrulewidth}{0.4pt}		% Linea cabeza de pagina
\renewcommand{\footrulewidth}{0.4pt}		% Linea pie de pagina

\begin{document}
\pagenumbering{gobble}

% Pagina de titulo
\begin{titlepage}

\begin{minipage}{\textwidth}

\centering

\includegraphics[scale=0.5]{img/ugr.png}\\

\textsc{\Large \asignatura{}\\[0.2cm]}
\textsc{GRADO EN INGENIERÍA INFORMÁTICA}\\[1cm]

\noindent\rule[-1ex]{\textwidth}{1pt}\\[1.5ex]
\textsc{{\Huge \titulo\\[0.5ex]}}
\textsc{{\Large \subtitulo\\}}
\noindent\rule[-1ex]{\textwidth}{2pt}\\[3.5ex]

\end{minipage}

\vspace{0.5cm}

\begin{minipage}{\textwidth}

\centering

\textbf{Autor}\\ {\autor{}}\\[2.5ex]
\textbf{Rama}\\ {Computación y Sistemas Inteligentes || Grupo 2 - Pablo Mesejo}\\[2.5ex]
\vspace{0.3cm}

\includegraphics[scale=0.3]{img/etsiit.jpeg}

\vspace{0.7cm}
\textsc{Escuela Técnica Superior de Ingenierías Informática y de Telecomunicación}\\
\vspace{1cm}
\textsc{Curso 2019-2020}
\end{minipage}
\end{titlepage}

\pagenumbering{arabic}
\tableofcontents
\thispagestyle{empty}				% No usar estilo en la pagina de indice

\newpage

\setlength{\parskip}{1em}
\setcounter{page}{1}

\section*{Introducción}
En esta práctica tendremos que realizar varias tareas en el mundo de \textit{StarCraft}, confeccionando un dominio de planificación
clásico mediante el lenguaje PDDL. Estos ficheros se ejecutarán en el planificador \textit{Metric-FF} que nos sacará un plan válido.

Se dispondrá de un par de ficheros por cada ejercicio (dominio y problema), los cuales ejecutaremos con la siguiente orden:
\vspace{-8px}
\begin{center}
    \texttt{./ff -o <ruta-dominio> -f <ruta-problema> -s 0}
\end{center}

El código para cada uno de los ejercicios es incremental con respecto al anterior, es decir, voy metiendo pequeñas modificaciones
dependiendo de lo que se nos pida. Esto no sucede en el último ejercicio, ya que he tenido que cambiar varias cosas para que se
ejecute en un tiempo razonable. Los cambios los explicaré más detalladamente después.


\section*{Ejercicio 1}
\addcontentsline{toc}{section}{Ejercicio 1}
En este ejercicio se nos pide que implementemos todo el dominio de \textit{StarCraft}, es decir, edificios, unidades, recursos y
acciones básicas como moverse, asignar o construir edificios.

La declaración de los tipos, las constantes y predicados no tiene mucho misterio, asi que viendo el código puede verse fácilmente
cómo y por qué están declaradas. En cuanto a las acciones si las vamos a explicar mejor:

\begin{itemize}
    \item \textbf{Navegar.} Mueve una unidad entre una localización y otra. Para ello, primero comprueba si la localización en la
          que se encuentra está conectada con la localización a la que va; y despues se asegura que la unidad no estaba extrayendo
          ningún recurso. Si esto se cumple, añade la unidad a la nueva localización y la elimina de la anterior
    \item \textbf{Asignar.} Pone a un VCE a extraer recursos de un nodo. Si una unidad del tipo VCE se encuentra en la posición
          del recurso a extraer y no lo estaba extrayendo previamente, comienza a extraerlo.
    \item \textbf{Construir.} Ordena a un trabajador libre que construya un edificio. Al igual que antes, este trabajador tiene
          que ser un VCE, estar en la localización donde se va a construir y no puede estar ocupado extrayendo. Además, tenemos
          que comprobar que no hay ningún edificio ya construido en esa localización, y para ello hemos utilizado el predicado
          '\texttt{exists}', que busca en el resto de localizaciones ese edificio. Por último, también tiene que existir otra
          unidad distinta, extrayendo el recurso que necesita el edifico para poder construirse.
\end{itemize}

Por otro lado, tenemos el fichero problema, en el cual declaramos y colocamos en el mapa tanto las unidades, como los edificios,
como los minerales. Como el mundo está representado en forma de cuadrícula, también tendremos que definirla en el problema
(en mi caso la hice gracias a la herramienta del editor online).



\section*{Ejercicio 2}
\addcontentsline{toc}{section}{Ejercicio 2}


\section*{Ejercicio 3}
\addcontentsline{toc}{section}{Ejercicio 3}


\section*{Ejercicio 4}
\addcontentsline{toc}{section}{Ejercicio 4}


\section*{Ejercicio 5}
\addcontentsline{toc}{section}{Ejercicio 5}


\section*{Ejercicio 6}
\addcontentsline{toc}{section}{Ejercicio 6}


\end{document}