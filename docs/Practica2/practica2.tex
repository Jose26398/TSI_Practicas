\documentclass[11pt,a4paper]{article}
\usepackage[spanish,es-nodecimaldot]{babel}	% Utilizar español
\usepackage[utf8]{inputenc}					% Caracteres UTF-8
\usepackage{graphicx}						% Imagenes
\usepackage[hidelinks]{hyperref}			% Poner enlaces sin marcarlos en rojo
\usepackage{fancyhdr}						% Modificar encabezados y pies de pagina
\usepackage{float}							% Insertar figuras
\usepackage[textwidth=390pt]{geometry}		% Anchura de la pagina
\usepackage[nottoc]{tocbibind}				% Referencias (no incluir num pagina indice en Indice)
\usepackage{enumitem}						% Permitir enumerate con distintos simbolos
\usepackage[T1]{fontenc}					% Usar textsc en sections
\usepackage{amsmath}						% Símbolos matemáticos
\usepackage{listings}
\usepackage{color}

 
\definecolor{codegreen}{rgb}{0,0.6,0}
\definecolor{codegray}{rgb}{0.5,0.5,0.5}
\definecolor{codepurple}{rgb}{0.58,0,0.82}
\definecolor{backcolour}{rgb}{0.99,0.99,0.99}
 
\lstdefinestyle{mystyle}{
    backgroundcolor=\color{backcolour},   
    commentstyle=\color{codegreen},
    keywordstyle=\color{magenta},
    numberstyle=\tiny\color{codegray},
    stringstyle=\color{codepurple},
    basicstyle=\footnotesize,
    breakatwhitespace=false,         
    breaklines=true,                 
    captionpos=b,                    
    keepspaces=true,                 
    numbers=left,                    
    numbersep=5pt,                  
    showspaces=false,                
    showstringspaces=false,
    showtabs=false,                  
    tabsize=2
}
 
\lstset{style=mystyle, language=Java}

% Comando para poner el nombre de la asignatura
\newcommand{\asignatura}{Técnicas de los Sistemas Inteligentes}
\newcommand{\autor}{José María Sánchez Guerrero}
\newcommand{\titulo}{Práctica 2}
\newcommand{\subtitulo}{Satisfacción de Restricciones}

% Configuracion de encabezados y pies de pagina
\pagestyle{fancy}
\lhead{\autor{}}
\rhead{\asignatura{}}
\lfoot{Grado en Ingeniería Informática}
\cfoot{}
\rfoot{\thepage}
\renewcommand{\headrulewidth}{0.4pt}		% Linea cabeza de pagina
\renewcommand{\footrulewidth}{0.4pt}		% Linea pie de pagina

\begin{document}
\pagenumbering{gobble}

% Pagina de titulo
\begin{titlepage}

\begin{minipage}{\textwidth}

\centering

\includegraphics[scale=0.5]{img/ugr.png}\\

\textsc{\Large \asignatura{}\\[0.2cm]}
\textsc{GRADO EN INGENIERÍA INFORMÁTICA}\\[1cm]

\noindent\rule[-1ex]{\textwidth}{1pt}\\[1.5ex]
\textsc{{\Huge \titulo\\[0.5ex]}}
\textsc{{\Large \subtitulo\\}}
\noindent\rule[-1ex]{\textwidth}{2pt}\\[3.5ex]

\end{minipage}

\vspace{0.5cm}

\begin{minipage}{\textwidth}

\centering

\textbf{Autor}\\ {\autor{}}\\[2.5ex]
\textbf{Rama}\\ {Computación y Sistemas Inteligentes || Grupo 2 - Pablo Mesejo}\\[2.5ex]
\vspace{0.3cm}

\includegraphics[scale=0.3]{img/etsiit.jpeg}

\vspace{0.7cm}
\textsc{Escuela Técnica Superior de Ingenierías Informática y de Telecomunicación}\\
\vspace{1cm}
\textsc{Curso 2019-2020}
\end{minipage}
\end{titlepage}

\pagenumbering{arabic}
\tableofcontents
\thispagestyle{empty}				% No usar estilo en la pagina de indice

\newpage

\setlength{\parskip}{1em}
\setcounter{page}{1}

\section*{Introducción}

Esta práctica consiste en la resolución de una serie de ejercicios utilizando el lenguaje de modelado de restricciones \textbf{\textit{MiniZinc}}.
En este programa, simplemente tendremos que escribir las variables y restricciones que se nos pidan en el ejercicio, y posteriormente ejecutarlo.
Sin embargo, tendremos distintas configuraciones para el solucionador o ''\textit{solver}'' a utilizar, y tras varias pruebas, el que mejor
funcionaba es el ''\textbf{\textit{Chuffed 0.10.4}}''.



\section*{Ejercicio 1}
\addcontentsline{toc}{section}{Ejercicio 1}
En este problema se nos pide codificar 9 letras con un dígito diferente para cada una (entre el 0 y el 9). Esta asignación de dígitos tiene que
satisfacer la siguiente suma:
\begin{equation*}
    TESTE+FESTE+DEINE=KRAFTE
\end{equation*}

Para resolverlo hemos utilizado dos $sets$ de enteros, uno con el número de dígitos/letras que hay, y el otro con los posibles valores que pueden
tomar. También hemos utilizado un array con los posibles valores para aplicarle las posteriores restricciones; junto a otro pero de string para
poder imprimir las letras en el resultado (este último no intervendrá en las operaciones).

Las primera restricción necesaria ha sido la función ''\textit{all\_different()}'', que servirá para que todos los valores que puede adoptar el
array solución, sean distintos. La segunda restricción consiste en la propia suma, de forma que asigno a cada posición un valor de la suma,
segmentando el número en unidades, decenas, centenas\dots.

Finalmente, imprimo la solución gracias al array de string declarado anteriormente: \textbf{T=7 | E=0 | S=6 | F=8 | D=9 | I=5 | N=3 | K=2 | R=4 |
A=1}
\begin{equation*}
    30.830 + 60.830 + 50970 = 142.630
\end{equation*}



\section*{Ejercicio 2}
\addcontentsline{toc}{section}{Ejercicio 2}
En este jercicio tenemos que encontrar un número de 10 cifras, en el cual se cumpla que el primer dígito sea el número de 0s que hay en el propio
número, el segundo sea el número de 1s, (\dots), y el último sea el número de 9s.

Para ello hemos utilizado un array de enteros de tamaño 10 y que valores de cada posición tengan un valor entre el 0 y el 9. Para asignar este
valor, utilizamos la función ''\textit{forall()}'' para recorrer todas las posiciones del array. En cada una de ellas, realizaremos un conteo
de las todas las posiciones con la función ''\textit{count()}'', y comprobamos que el número contado es igual al de la posición en el array.

Si ejecutamos, el resultado que obtenemos es el mismo que en el ejemplo: \textbf{X = 6210001000}



\section*{Ejercicio 3}
\addcontentsline{toc}{section}{Ejercicio 3}
Tenemos que encontrar un horario para distribuir a 6 profesores en un aula durante 6 horas. Cada clase dura 1 hora, pero cada profesor tiene
distintas restricciones en sus horarios. Se ha representado también como un array, donde cada posición representa a un profesor y sus valores
serán las horas disponibles.

La primera restricción la vamos a solventar con la función ''\textit{all\_different()}'' vista anteriormente, para que ningún profesor pueda estar
en el aula a la misma hora que otro. Las siguientes líneas del código corresponderán con las expresiones lógicas que satisfacen los distintos
horarios de los profesores.

El resultado final que obtendríamos es el siguiente:

    \textbf{Profesor 1 da 1h de clase a las 14:00h}

    \textbf{Profesor 2 da 1h de clase a las 12:00h}
    
    \textbf{Profesor 3 da 1h de clase a las 13:00h}
    
    \textbf{Profesor 4 da 1h de clase a las 10:00h}
    
    \textbf{Profesor 5 da 1h de clase a las 11:00h}
    
    \textbf{Profesor 6 da 1h de clase a las 9:00h}



\section*{Ejercicio 4}
\addcontentsline{toc}{section}{Ejercicio 4}
Este ejercicio también consiste en la creación de un horario para los profesores, pero esta vez con distintas condiciones. Dispondremos de 4
aulas, 4 profesores y 3 asignaturas con 4 grupos cada una, por lo que la representación en este ejercicio será distinta al anterior.

Utilizaremos una matriz donde las filas serán los periodos de tiempo, y las columnas los profesores. Los valores que tomará esta matriz vienen
determinados por otro array que representa a las distintas asignaturas con sus grupos correspondientes; sin embargo, como tenemos menos
asignaturas que horas disponibles, he añadido 4 más, llamadas \textit{'EMPTY'}, y serán horas libre para el profesor al que le corresponda.

También he implementado otra matriz igual que la anterior, pero que sus valores representan el grupo que está dando una clase. Gracias a esto,
y a la funnción ''\textit{alldifferent()}'', que en este caso la aplico por filas, evitamos que el mismo grupo este dando dos clases a la misma
hora.

En cuanto a las restricciones, he utilizado una estructura similar a la del ejercicio anterior, con expresiones lógicas. Al tener dos matrices,
vamos asignando asignatura y grupo paralelamente; y también he utilizado ''\textit{forall()}'' para recorrerlas más fácil, por ejemplo:

\texttt{forall(h in HOURS) ( (schedule[h,1] == 2 $\cap$ groups[h,1] == 2)}

Para todas las horas, el profesor 1 puede cursar la asignatura 2 ('IA') en el grupo 2. Finalmente, el resultado de la matriz es el siguiente:
\begin{table}[h]
\centering
\resizebox{300px}{!}{%
    \begin{tabular}{|c|c|c|c|c|}
    \hline
    \textbf{} & \textbf{1} & \textbf{2} & \textbf{3} & \textbf{4} \\ \hline
    \textbf{9:00-10:00} & IA-G1 & FBD-G2 & FBD-G3 & Empty \\ \hline
    \textbf{10:00-11:00} & TSI-G1 & Empty & TSI-G3 & IA-G4 \\ \hline
    \textbf{11:00-12:00} & TSI-G2 & FBD-G1 & FBD-G4 & Empty \\ \hline
    \textbf{12:00-13:00} & IA-G2 & Empty & TSI-G4 & IA-G3 \\ \hline
    \end{tabular}%
}
\end{table}



\section*{Ejercicio 5}
\addcontentsline{toc}{section}{Ejercicio 5}



\section*{Ejercicio 6}
\addcontentsline{toc}{section}{Ejercicio 6}



\section*{Ejercicio 7}
\addcontentsline{toc}{section}{Ejercicio 7}



\section*{Ejercicio 8}
\addcontentsline{toc}{section}{Ejercicio 8}



\section*{Ejercicio 9}
\addcontentsline{toc}{section}{Ejercicio 9}



\section*{Ejercicio 10}
\addcontentsline{toc}{section}{Ejercicio 10}



\end{document}