\documentclass[11pt,a4paper]{article}
\usepackage[spanish,es-nodecimaldot]{babel}	% Utilizar español
\usepackage[utf8]{inputenc}					% Caracteres UTF-8
\usepackage{graphicx}						% Imagenes
\usepackage[hidelinks]{hyperref}			% Poner enlaces sin marcarlos en rojo
\usepackage{fancyhdr}						% Modificar encabezados y pies de pagina
\usepackage{float}							% Insertar figuras
\usepackage[textwidth=390pt]{geometry}		% Anchura de la pagina
\usepackage[nottoc]{tocbibind}				% Referencias (no incluir num pagina indice en Indice)
\usepackage{enumitem}						% Permitir enumerate con distintos simbolos
\usepackage[T1]{fontenc}					% Usar textsc en sections
\usepackage{amsmath}						% Símbolos matemáticos
\usepackage{listings}
\usepackage{color}

 
\definecolor{codegreen}{rgb}{0,0.6,0}
\definecolor{codegray}{rgb}{0.5,0.5,0.5}
\definecolor{codepurple}{rgb}{0.58,0,0.82}
\definecolor{backcolour}{rgb}{0.99,0.99,0.99}
 
\lstdefinestyle{mystyle}{
    backgroundcolor=\color{backcolour},   
    commentstyle=\color{codegreen},
    keywordstyle=\color{magenta},
    numberstyle=\tiny\color{codegray},
    stringstyle=\color{codepurple},
    basicstyle=\footnotesize,
    breakatwhitespace=false,         
    breaklines=true,                 
    captionpos=b,                    
    keepspaces=true,                 
    numbers=left,                    
    numbersep=5pt,                  
    showspaces=false,                
    showstringspaces=false,
    showtabs=false,                  
    tabsize=2
}
 
\lstset{style=mystyle, language=Java}

% Comando para poner el nombre de la asignatura
\newcommand{\asignatura}{Técnicas de los Sistemas Inteligentes}
\newcommand{\autor}{José María Sánchez Guerrero}
\newcommand{\titulo}{Práctica 2}
\newcommand{\subtitulo}{Satisfacción de Restricciones}

% Configuracion de encabezados y pies de pagina
\pagestyle{fancy}
\lhead{\autor{}}
\rhead{\asignatura{}}
\lfoot{Grado en Ingeniería Informática}
\cfoot{}
\rfoot{\thepage}
\renewcommand{\headrulewidth}{0.4pt}		% Linea cabeza de pagina
\renewcommand{\footrulewidth}{0.4pt}		% Linea pie de pagina

\begin{document}
\pagenumbering{gobble}

% Pagina de titulo
\begin{titlepage}

\begin{minipage}{\textwidth}

\centering

\includegraphics[scale=0.5]{img/ugr.png}\\

\textsc{\Large \asignatura{}\\[0.2cm]}
\textsc{GRADO EN INGENIERÍA INFORMÁTICA}\\[1cm]

\noindent\rule[-1ex]{\textwidth}{1pt}\\[1.5ex]
\textsc{{\Huge \titulo\\[0.5ex]}}
\textsc{{\Large \subtitulo\\}}
\noindent\rule[-1ex]{\textwidth}{2pt}\\[3.5ex]

\end{minipage}

\vspace{0.5cm}

\begin{minipage}{\textwidth}

\centering

\textbf{Autor}\\ {\autor{}}\\[2.5ex]
\textbf{Rama}\\ {Computación y Sistemas Inteligentes || Grupo 2 - Pablo Mesejo}\\[2.5ex]
\vspace{0.3cm}

\includegraphics[scale=0.3]{img/etsiit.jpeg}

\vspace{0.7cm}
\textsc{Escuela Técnica Superior de Ingenierías Informática y de Telecomunicación}\\
\vspace{1cm}
\textsc{Curso 2019-2020}
\end{minipage}
\end{titlepage}

\pagenumbering{arabic}
\tableofcontents
\thispagestyle{empty}				% No usar estilo en la pagina de indice

\newpage

\setlength{\parskip}{1em}
\setcounter{page}{1}

\section*{Introducción}

Esta práctica consiste en la resolución de una serie de ejercicios utilizando el lenguaje de modelado de restricciones \textbf{\textit{MiniZinc}}.
En este programa, simplemente tendremos que escribir las variables y restricciones que se nos pidan en el ejercicio, y posteriormente ejecutarlo.
Sin embargo, tendremos distintas configuraciones para el solucionador o ''\textit{solver}'' a utilizar, y tras varias pruebas, el que mejor
funcionaba es el ''\textbf{\textit{COIN-BC 2.10}}''.



\section*{Ejercicio 1}
\addcontentsline{toc}{section}{Ejercicio 1}
En este problema se nos pide codificar 9 letras con un dígito diferente para cada una (entre el 0 y el 9). Esta asignación de dígitos tiene que
satisfacer la siguiente suma:
\begin{equation*}
    TESTE+FESTE+DEINE=KRAFTE
\end{equation*}

Para resolverlo hemos utilizado dos $sets$ de enteros, uno con el número de dígitos/letras que hay, y el otro con los posibles valores que pueden
tomar. También hemos utilizado un array con los posibles valores para aplicarle las posteriores restricciones; junto a otro pero de string para
poder imprimir las letras en el resultado (este último no intervendrá en las operaciones).

Las primera restricción necesaria ha sido la función ''\textit{all\_different()}'', que servirá para que todos los valores que puede adoptar el
array solución, sean distintos. La segunda restricción consiste en la propia suma, de forma que asigno a cada posición un valor de la suma,
segmentando el número en unidades, decenas, centenas\dots.

Finalmente, imprimo la solución gracias al array de string declarado anteriormente: \textbf{T=7 | E=0 | S=6 | F=8 | D=9 | I=5 | N=3 | K=2 | R=4 |
A=1}
\begin{equation*}
    30.830 + 60.830 + 50970 = 142.630
\end{equation*}



\section*{Ejercicio 2}
\addcontentsline{toc}{section}{Ejercicio 2}
En este jercicio tenemos que encontrar un número de 10 cifras, en el cual se cumpla que el primer dígito sea el número de 0s que hay en el propio
número, el segundo sea el número de 1s, (\dots), y el último sea el número de 9s.

Para ello hemos utilizado un array de enteros de tamaño 10 y que valores de cada posición tengan un valor entre el 0 y el 9. Para asignar este
valor, utilizamos la función ''\textit{forall()}'' para recorrer todas las posiciones del array. En cada una de ellas, realizaremos un conteo
de las todas las posiciones con la función ''\textit{count()}'', y comprobamos que el número contado es igual al de la posición en el array.

Si ejecutamos, el resultado que obtenemos es el mismo que en el ejemplo: \textbf{X = 6210001000}



\section*{Ejercicio 3}
\addcontentsline{toc}{section}{Ejercicio 3}
Tenemos que encontrar un horario para distribuir a 6 profesores en un aula durante 6 horas. Cada clase dura 1 hora, pero cada profesor tiene
distintas restricciones en sus horarios. Se ha representado también como un array, donde cada posición representa a un profesor y sus valores
serán las horas disponibles.

La primera restricción la vamos a solventar con la función ''\textit{all\_different()}'' vista anteriormente, para que ningún profesor pueda estar
en el aula a la misma hora que otro. Las siguientes líneas del código corresponderán con las expresiones lógicas que satisfacen los distintos
horarios de los profesores.

El resultado final que obtendríamos es el siguiente:

    \textbf{Profesor 1 da 1h de clase a las 14:00h}

    \textbf{Profesor 2 da 1h de clase a las 12:00h}
    
    \textbf{Profesor 3 da 1h de clase a las 13:00h}
    
    \textbf{Profesor 4 da 1h de clase a las 10:00h}
    
    \textbf{Profesor 5 da 1h de clase a las 11:00h}
    
    \textbf{Profesor 6 da 1h de clase a las 9:00h}



\section*{Ejercicio 4}
\addcontentsline{toc}{section}{Ejercicio 4}
Este ejercicio también consiste en la creación de un horario para los profesores, pero esta vez con distintas condiciones. Dispondremos de 4
aulas, 4 profesores y 3 asignaturas con 4 grupos cada una, por lo que la representación en este ejercicio será distinta al anterior.

Utilizaremos una matriz donde las filas serán los periodos de tiempo, y las columnas los profesores. Los valores que tomará esta matriz vienen
determinados por otro array que representa a las distintas asignaturas con sus grupos correspondientes; sin embargo, como tenemos menos
asignaturas que horas disponibles, he añadido 4 más, llamadas \textit{'EMPTY'}, y serán horas libre para el profesor al que le corresponda.

También he implementado otra matriz igual que la anterior, pero que sus valores representan el grupo que está dando una clase. Gracias a esto,
y a la funnción ''\textit{alldifferent()}'', que en este caso la aplico por filas, evitamos que el mismo grupo este dando dos clases a la misma
hora.

En cuanto a las restricciones, he utilizado una estructura similar a la del ejercicio anterior, con expresiones lógicas. Al tener dos matrices,
vamos asignando asignatura y grupo paralelamente; y también he utilizado ''\textit{forall()}'' para recorrerlas más fácil, por ejemplo:

\texttt{forall(h in HOURS) ( (schedule[h,1] == 2 $\cap$ groups[h,1] == 2)}

Para todas las horas, el profesor 1 puede cursar la asignatura 2 ('IA') en el grupo 2. Finalmente, el resultado de la matriz es el siguiente:
\begin{table}[h]
\centering
\resizebox{300px}{!}{%
    \begin{tabular}{|c|c|c|c|c|}
    \hline
    \textbf{} & \textbf{1} & \textbf{2} & \textbf{3} & \textbf{4} \\ \hline
    \textbf{9:00-10:00} & IA-G1 & FBD-G2 & FBD-G3 & Empty \\ \hline
    \textbf{10:00-11:00} & TSI-G1 & Empty & TSI-G3 & IA-G4 \\ \hline
    \textbf{11:00-12:00} & TSI-G2 & FBD-G1 & FBD-G4 & Empty \\ \hline
    \textbf{12:00-13:00} & IA-G2 & Empty & TSI-G4 & IA-G3 \\ \hline
    \end{tabular}%
}
\end{table}



\section*{Ejercicio 5}
\addcontentsline{toc}{section}{Ejercicio 5}
Este ejercicio es muy similar al ejercicio anterior. Tenemos que encontrar una asignación de horarios que satisfagan una serie de condiciones.
Lo que cambia en este ejercicio son las asignaturas y el tiempo, ya que tendremos que cumplir un número de horas semanales, impartirlas
por bloques. Tabién tendremos más restricciones en cuanto a profesores u horas que tienen que estar libres.

Lo primero que haremos es definir las variables horas y días, y los arrays asignaturas, las horas por asignatura a la semana, y el profesor que
imparte cada asignatura. Después creamos dos matrices que representen el horario semanal (horas como filas y días como columnas), uno para las
asignaturas y otro para los profesores, los cuales rellenaremos con los valores de la matriz asignatura y el número del profesor, respectivamente.

Las primeras restricciones serán dos ''\textit{all\_different}'' para evitar que en un dia se de más de una clase del mismo bloque, y el otro
para evitar que un profesor imparta más de un bloque al día. A continuación, ponemos que el cuarto periodo sea el recreo para todos los días, y
que cada una de las asignaturas imparta sus horas correspondientes gracias al array declarado previamente.

Pasamos a poner las restricciones de horarios. Empezamos poniendo que las clases que duran 2 horas no puedan emepzar antes del recreo o antes
de irse a casa. Para ello simplemente recorremos la matriz por días y con una expresión lógica decimos que esas horas no puedan ser iguales a
las asignaturas correspondientes. La otra restricción de horario que hay que tener en cuenta es que cuando comience una asignatura de 2 horas,
la siguiente sea esta misma asignatura. En nuestro caso, para facilitarnos las cosas, hemos decidido poner en el propio array de las asignaturas
una asignatura AX - parte 2, por ejemplo, \textit{A1-2}. Gracias a esto, recorriendo todos los valores de la matriz (excepto los últimos y los
recreos) y asignamos las segundas partes correspondientes.

Por último nos queda la restricción de los profesores, para la cual utilizamos la otra matriz creada al principio. Ésta la rellenaremos gracias
a la matriz de las asignaturas ya obtenida, y dependiendo de que valores tengamos en ella, asignamos un valor u otro a la de profesores. Aclarar
que no se asignan profesores, si no valores. Esto lo hacemos para tener un profesor distinto gracias a las primeras restricciones. Las
condiciones implementadas en el código  nos servirán para poner el mismo valor a las asignaturas dadas por el mismo profesor.

El resultado obtenido tras ejecutar es el siguiente:
\begin{table}[H]
    \centering
    \resizebox{380px}{!}{%
    \begin{tabular}{|c|c|l|l|l|l|}
    \hline
     & \textbf{1} & \multicolumn{1}{c|}{\textbf{2}} & \multicolumn{1}{c|}{\textbf{3}} & \multicolumn{1}{c|}{\textbf{4}} & \multicolumn{1}{c|}{\textbf{5}} \\ \hline
    \textbf{8:00-9:00} & A1 (Prof:1) & \multicolumn{1}{c|}{A4 (Prof:2)} & A9 (Prof:3) & A5 (Prof:2) & A3 (Prof:1) \\ \hline
    \textbf{9:00-10:00} & A1 (Prof:1) & A4 (Prof:2) & A2 (Prof:4) & A5 (Prof:2) & A3 (Prof:1) \\ \hline
    \textbf{10:00-11:00} & A6 (Prof:3) & A7 (Prof:4) & A7 (Prof:4) & A6 (Prof:3) & A2 (Prof:4) \\ \hline
    \textbf{11:00-12:00} & Recreo & \multicolumn{1}{c|}{Recreo} & \multicolumn{1}{c|}{Recreo} & \multicolumn{1}{c|}{Recreo} & \multicolumn{1}{c|}{Recreo} \\ \hline
    \textbf{12:00-13:00} & A5 (Prof:2) & A1 (Prof:1) & A3 (Prof:1) & A8 (Prof:4) & A4 (Prof:2) \\ \hline
    \textbf{13:00-14:00} & A5 (Prof:2) & A1 (Prof:1) & A3 (Prof:1) & A8 (Prof:4) & A4 (Prof:2) \\ \hline
    \end{tabular}%
    }
    \end{table}



\section*{Ejercicio 6}
\addcontentsline{toc}{section}{Ejercicio 6}
Ahora se nos plantea un problema que consiste en adivinar dónde está la cebra y qué persona bebe agua en un vecindario. Este vecindario esta
formado por 5 personas, las cuales tienen una serie de características que las diferencian unas de otras: región, profesión, animal, bebida,
color de la casa y posición de la casa con respecto a los demás (están en una línea recta).

La forma en la que se ha resuelto el problema es muy sencilla. He creado una matriz de tamaño $5x6$ (número de vecinos x características) y
cada uno de los valores que tomará una celda vienen definidos por lo siguiente. Las características serán arrays de strings con todas las
características disponibles, asi que cada valor de la matriz será la posición (int) que tiene esa característica en el array.

Una vez hecho esto, lo siguiente será simplemente poner las expresiones lógicas correspondientes:

\texttt{\% El vasco vive en la casa roja.}

\vspace{-8px}

\texttt{constraint matrix[1,4] == 1;}

siendo la primera fila de la matriz el vasco, la cuarta columna el color de la casa y '\textit{colors[\textbf{1}]}' el rojo. El resultado final
es el siguiente:

\textbf{El vasco es escultor, tiene caracoles, bebe leche, su casa en roja y vive en la posicion centro}

\textbf{El catalan es violinista, tiene perro, bebe zumo, su casa en blanca y vive en la posicion  centroderecha}

\textbf{El gallego es pintor, tiene cebra, bebe cafe, su casa en verde y vive en la posicion derecha}

\textbf{El navarro es medico, tiene caballo, bebe te, su casa en azul y vive en la posicion centroizquierda}

\textbf{El andaluz es diplomatico, tiene zorro, bebe agua, su casa en amarilla y vive en la posicion izquierda}

Por lo tanto, la \textbf{CEBRA} la tiene el gallego, y el andaluz bebe \textbf{AGUA}.


\section*{Ejercicio 7}
\addcontentsline{toc}{section}{Ejercicio 7}
En este ejercicio disponemos de la informaciónn necesaria para construir una casa paso a paso. Viene representada en una tabla, donde la primera
columna son los identificadores de las tareas necesarias para construirla, la segunda sus nombres, la tercera el tiempo que se tarda en hacerla
y la cuarta las tareas que tienen que estar hechas antes de realizar la actual (relación de precedencia).

Empezamos declarando un array de string con todas las tareas que llevaremos a cabo, y otro con la duración de cada una de ellas. A continuación,
guardamos en una variable el tiempo máximo que pueden tardar las tareas (si las ejecutamos una tras otra).

También tenemos una matriz de $9x2$ para las precedencias, donde la primera columna es la tarea actual y la segunda es la tarea que no se puede
ejecutar hasta que termine. El resto de variables que declaramos son los tiempos en los que empieza a construir (\texttt{start})(ponemos 0
porque el día 1 ya lo estaríamos contando como un día de trabajo), los tiempos en los que finalizan las tareas (\texttt{end}), y el tiempo total
que lleva construyéndose la casa.

Ahora vamos a pasar a las restricciones utilizadas. La primera de ellas es la función ''\texttt{maximum()}'', y con la cual nos quedamos siempre
con el valor máximo que tiene la variable '\texttt{end}'.

Pasamos a la función ''\texttt{cumulative()}'', que es la más importante en este ejercicio. Ésta acepta como parámetros nuestros arrays 'start'
y el de duración de cada tarea, y con ellos calcular el tiempo que tardarían una serie de trabjadores en realizarlas. Estos trabjadores se
declaran en los siguientes parámetros, pero como en nuestro caso tenemos siempre uno disponible, se ha asignado uno para cada tarea y 1000
disponibles (un número ilimitado de éstos).

Por último nos queda hacer dos ''\texttt{forall()}''. Uno que recorra las tareas y vaya sumando a la variable \texttt{end} la suma del inicio
más la duración; y otro que recorra las precedencias y evite que se ejecuten las que tienen un tiempo más alto que sus predecesoras. También
nos queda comentar que, como se nos dice el enunciado, utilizamos la función ''\textbf{solve minimize total\_time}'' con la variable del tiempo
máximo obtenido antes.

El resultado final es el siguiente:

\textbf{Tiempo total: 18} \vspace{-8px}

\textbf{Levantar\_muros: empieza el dia 0, tarda 7 (termina el dia 7)} \vspace{-8px}

\textbf{Carpinteria\_de\_tejado: empieza el dia 7, tarda 3 (termina el dia 10)}\vspace{-8px}

\textbf{Tejado : empieza el dia 11, tarda 1 (termina el dia 12)} \vspace{-8px}

\textbf{Instalacion\_electrica : empieza el dia 7, tarda 8 (termina el dia 15)} \vspace{-8px}

\textbf{Pintado\_fachada : empieza el dia 16, tarda 2 (termina el dia 18)} \vspace{-8px}

\textbf{Ventanas : empieza el dia 15, tarda 1 (termina el dia 16)} \vspace{-8px}

\textbf{Jardin : empieza el dia 17, tarda 1 (termina el dia 18)} \vspace{-8px}

\textbf{Techado : empieza el dia 8, tarda 3 dias (termina el dia 11)} \vspace{-8px}

\textbf{Pintado\_interior : empieza el dia 16, tarda 2 dias (termina el dia 18)} \vspace{-8px}



\section*{Ejercicio 8}
\addcontentsline{toc}{section}{Ejercicio 8}
Este ejercicio es exactamente igual que el anterior, con el único cambio de que esta vez sí que tenemos un número de trabajadores limitados (3) y
además cada una de las tareas requiere una cantidad de trabajadores determinada.

Para solventar esta nueva restricción, añadimos una variable para los trabajadores disponiobles y otro array con los necesarios para realizar
una determinada tarea. Estos tiempos son los que están en la tabla del enunciado. Como dijimos en el ejercicio anterior, la función encargada
de manejar estos tiempos es la función ''\texttt{cumulative()}'', en la que pusimos infinitos trabajadores y que cada tarea sólo necesitaba
a uno. Pues en este caso, en el parámetro que nos dice los trabajdores necesarios por tarea meteremos el array, y en el siguiente los trabajdores
disponibles (3 en este caso).

\texttt{cumulative(start, duration, workers, available\_workers)}

El resto del programa se mantendría exactamente igual. El resultado final si ejecutamos es el siguiente:

\textbf{Tiempo total: 22} \vspace{-8px}

\textbf{Levantar\_muros: empieza el dia 0, tarda 7 (termina el dia 7)} \vspace{-8px}

\textbf{Carpinteria\_de\_tejado: empieza el dia 7, tarda 3 (termina el dia 10)}\vspace{-8px}

\textbf{Tejado : empieza el dia 18, tarda 1 (termina el dia 19)} \vspace{-8px}

\textbf{Instalacion\_electrica : empieza el dia 10, tarda 8 (termina el dia 18)}\vspace{-8px}

\textbf{Pintado\_fachada : empieza el dia 20, tarda 2 (termina el dia 22)} \vspace{-8px}

\textbf{Ventanas : empieza el dia 19, tarda 1 (termina el dia 20)} \vspace{-8px}

\textbf{Jardin : empieza el dia 19, tarda 1 (termina el dia 20)} \vspace{-8px}

\textbf{Techado : empieza el dia 12, tarda 3 dias (termina el dia 15)} \vspace{-8px}

\textbf{Pintado\_interior : empieza el dia 20, tarda 2 dias (termina el dia 22)} \vspace{-8px}



\section*{Ejercicio 9}
\addcontentsline{toc}{section}{Ejercicio 9}



\section*{Ejercicio 10}
\addcontentsline{toc}{section}{Ejercicio 10}



\end{document}